Kad se govori o prenošenju stila glazbe, podrazumijeva se podijeliti glazbeni primjerak na njegov sadržaj i njegov stil. Cilj nam je izmijeniti taj stil u neki drugi zadani stil, a da sadržaj primjerka ostane uvelike nepromijenjen. U te se svrhe koriste autoregresijski modeli, generativne suparničke mreže (engl. \textit{Generative Adversarial Networks - GAN})\cite{goodfellow2020generative}, varijacijski autoenkoderi\cite{kingma2019introduction} i razni drugi modeli. U području GAN-ova ističu se modifikacije WaveGAN\cite{donahue2019adversarialaudiosynthesis} i CycleGAN\cite{brunner2018symbolicmusicgenretransfer}. \textit{Score Matching} još je jedna nedavna metoda korištena u generiranju glazbenih zapisa\cite{song2021scorebasedgenerativemodelingstochastic}, no rezultati se obično fokusiraju na prenošenje samo jedne karakteristike stila i pate od artefakata na spektrogramima.

Neki nedavni radovi poput WaveGrad\cite{chen2020wavegradestimatinggradientswaveform} i DiffWave\cite{kong2021diffwaveversatilediffusionmodel} uspijevaju generirati audio podatke vrlo brzo i kvalitetno pomoću difuzijskih modela. Glavni uzor u tom polju ipak nam je ranije spomenut rad\cite{huang2024musicstyletransferdiffusion} u kojem autori opisuju karakteristike korištene podatkovne reprezentacije, korištenje latentnog difuzijskog modela za učenje i prenošenje zadanog stila glazbe, kao i dodavanje vremenski promjenjivog kodera u arhitekturu kako bi se dodatno unaprijedilo prenošenje stila.

Valja napomenuti i da se glavnina tradicionalnih modela prenošenja stila bavi prebacivanjem pjesama u druge striktno definirane glazbene žanrove, dok su mnogi stilovi na kojima se naš model uči manje instrumentalni. Neki od njih su zvukovi pucketanja plamena, cvrkuta ptice, zvonjave zvona ili otkucaja srca.